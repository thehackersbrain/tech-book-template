\chapter{Gentoo Installation}

\section{Why Necessary?}
So most of the times, linux-firmware, initramfs, and kernel itself works hand in hand, but if the kernel is highly customized and configured correctly, then all of these un-necessary packages like linux-firmware and initramfs can be avoided. then comes together to support and fit as different systems as a whole?

Yes, you’ve got the essence of it! Here's how it works in more detail:

\section{The Relationship and The Kernel}

\begin{itemize}
    \item \textbf{Kernel}: The core of the system, providing the basic support of all hardware (CPU, NIC, sound card, etc.). The kernel can be customized and configured in many ways to support specific hardware directly.
    \item \textbf{linux-firmware}: Provides binary blobs (firmware) required by certain hardware devices that aren't built into the kernel. For example, many wireless cards, GPUs, and other specialized devices need this extra firmware.
    \item \textbf{initramfs}: A temporary root filesystem loaded into memory during boot, allowing the kernel to initialize hardware, load necessary modules, and access the real root filesystem. It’s especially useful when you have encrypted filesystems, complex RAID configurations, or hardware that requires modules not built into the kernel.
\end{itemize}

\begin{lstlisting}[caption={Sample Code}, label={lst:sample}]
def hello_world():
    print("Hello, world!")
\end{lstlisting}

\section{How They Work Together}
\begin{enumerate}
  \item \textbf{Kernel and Linux Firmware}:
    \begin{itemize}
        \item The kernel provides drivers to interact with hardware, but many devices (especially wireless cards, GPUs, etc.) require firmware that is not part of the kernel.
        \item If your hardware is supported by the kernel and doesn't require any additional proprietary firmware, you don’t need to worry about linux-firmware.
        \item But if you have hardware like certain Wi-Fi or Bluetooth cards, linux-firmware will be necessary to load the required binary blobs during boot\footnote{This is a footnote}.
    \end{itemize}
\item Kernel and initramfs:
    \begin{itemize}
        \item The kernel can directly access hardware if all drivers are built into it (no need for additional modules).
        \item If the hardware or system setup (e.g., encrypted partitions, RAID arrays, or non-standard root devices) needs additional modules, drivers, or other initializations, then initramfs will be required to load those things before the main filesystem is accessible.
        \item Without initramfs, the kernel might not be able to load certain drivers or modules that are required to boot, especially if those modules aren’t built directly into the kernel but are instead loaded at boot time.
    \end{itemize}
\end{enumerate}

\section{Highly Customized Kernel:}
If you \textit{customize your kernel} thoroughly by:
\begin{itemize}
  \item \textit{Including all necessary drivers directly in the kernel}, instead of as modules, and
  \item Ensuring that \textit{your hardware doesn't require any extra firmware},
\end{itemize}

then \textit{you can avoid both} \textbf{linux-firmware} and \textbf{initramfs}. The kernel will handle all the device initialization on its own, without needing \textit{initramfs} to load modules or extra firmware at boot time.

\newpage

\texttt{this is inline code}
\begin{lstlisting}[caption={Sample Code}, label={lst:sample}]
┌──(elliot@archlinux) - [~] - [E 192.168.1.2] - [10001]
└─[$] lspci                                                                                                                                    [1:06:15]
00:00.0 Host bridge: Advanced Micro Devices, Inc. [AMD] Renoir/Cezanne Root Complex
00:00.2 IOMMU: Advanced Micro Devices, Inc. [AMD] Renoir/Cezanne IOMMU
00:01.0 Host bridge: Advanced Micro Devices, Inc. [AMD] Renoir PCIe Dummy Host Bridge
00:02.0 Host bridge: Advanced Micro Devices, Inc. [AMD] Renoir PCIe Dummy Host Bridge
00:02.1 PCI bridge: Advanced Micro Devices, Inc. [AMD] Renoir/Cezanne PCIe GPP Bridge
00:02.2 PCI bridge: Advanced Micro Devices, Inc. [AMD] Renoir/Cezanne PCIe GPP Bridge
00:02.4 PCI bridge: Advanced Micro Devices, Inc. [AMD] Renoir/Cezanne PCIe GPP Bridge
00:08.0 Host bridge: Advanced Micro Devices, Inc. [AMD] Renoir PCIe Dummy Host Bridge
00:08.1 PCI bridge: Advanced Micro Devices, Inc. [AMD] Renoir Internal PCIe GPP Bridge to Bus
00:08.2 PCI bridge: Advanced Micro Devices, Inc. [AMD] Renoir Internal PCIe GPP Bridge to Bus
00:14.0 SMBus: Advanced Micro Devices, Inc. [AMD] FCH SMBus Controller (rev 51)
00:14.3 ISA bridge: Advanced Micro Devices, Inc. [AMD] FCH LPC Bridge (rev 51)
00:18.0 Host bridge: Advanced Micro Devices, Inc. [AMD] Cezanne Data Fabric; Function 0
00:18.1 Host bridge: Advanced Micro Devices, Inc. [AMD] Cezanne Data Fabric; Function 1
00:18.2 Host bridge: Advanced Micro Devices, Inc. [AMD] Cezanne Data Fabric; Function 2
00:18.3 Host bridge: Advanced Micro Devices, Inc. [AMD] Cezanne Data Fabric; Function 3
00:18.4 Host bridge: Advanced Micro Devices, Inc. [AMD] Cezanne Data Fabric; Function 4
00:18.5 Host bridge: Advanced Micro Devices, Inc. [AMD] Cezanne Data Fabric; Function 5
00:18.6 Host bridge: Advanced Micro Devices, Inc. [AMD] Cezanne Data Fabric; Function 6
00:18.7 Host bridge: Advanced Micro Devices, Inc. [AMD] Cezanne Data Fabric; Function 7
01:00.0 Ethernet controller: Realtek Semiconductor Co., Ltd. RTL8111/8168/8211/8411 PCI Express Gigabit Ethernet Controller (rev 15)
02:00.0 Network controller: Realtek Semiconductor Co., Ltd. RTL8852BE PCIe 802.11ax Wireless Network Controller
03:00.0 Non-Volatile memory controller: Phison Electronics Corporation PS5019-E19 PCIe4 NVMe Controller (DRAM-less) (rev 01)
04:00.0 VGA compatible controller: Advanced Micro Devices, Inc. [AMD/ATI] Barcelo (rev c2)
04:00.1 Audio device: Advanced Micro Devices, Inc. [AMD/ATI] Renoir Radeon High Definition Audio Controller
04:00.2 Encryption controller: Advanced Micro Devices, Inc. [AMD] Family 17h (Models 10h-1fh) Platform Security Processor
04:00.3 USB controller: Advanced Micro Devices, Inc. [AMD] Renoir/Cezanne USB 3.1
04:00.4 USB controller: Advanced Micro Devices, Inc. [AMD] Renoir/Cezanne USB 3.1
04:00.5 Multimedia controller: Advanced Micro Devices, Inc. [AMD] ACP/ACP3X/ACP6x Audio Coprocessor (rev 01)
04:00.6 Audio device: Advanced Micro Devices, Inc. [AMD] Family 17h/19h HD Audio Controller
05:00.0 SATA controller: Advanced Micro Devices, Inc. [AMD] FCH SATA Controller [AHCI mode] (rev 81)
05:00.1 SATA controller: Advanced Micro Devices, Inc. [AMD] FCH SATA Controller [AHCI mode] (rev 81)
┌──(elliot@archlinux) - [~] - [E 192.168.1.2] - [10002]
└─[$]
\end{lstlisting}
