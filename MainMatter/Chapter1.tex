\chapter{Gentoo Installation}

\begin{chapterdesc}
Lorem ipsum dolor sit amet, consectetuer adipiscing elit. Ut purus elit, vestibulum ut, placerat ac, adipiscing vitae, felis. Curabitur dic- tum gravida mauris. Nam arcu libero, nonummy eget, consectetuer id, vulputate a, magna. Donec vehicula augue eu neque. Pellen- tesque habitant morbi tristique senectus et netus et malesuada fames ac turpis egestas. Mauris ut leo. Cras viverra metus rhoncus sem. Nulla et lectus vestibulum urna fringilla ultrices. Phasellus eu tel- lus sit amet tortor gravida placerat.
\end{chapterdesc}

\begin{note}
  This is a point to note here
\end{note}

\begin{warning}
  This is a warning
\end{warning}

\section{Why Necessary?}
So most of the times, linux-firmware, initramfs, and kernel itself works hand in hand, but if the kernel is highly customized and configured correctly, then all of these un-necessary packages like linux-firmware and initramfs can be avoided. then comes together to support and fit as different systems as a whole?

Yes, you’ve got the essence of it! Here's how it works in more detail:

\begin{itemize}
    \item \textbf{Kernel}: The core of the system, providing the basic support of all hardware (CPU, NIC, sound card, etc.). The kernel can be customized and configured in many ways to support specific hardware directly.
    \item \textbf{linux-firmware}: Provides binary blobs (firmware) required by certain hardware devices that aren't built into the kernel. For example, many wireless cards, GPUs, and other specialized devices need this extra firmware.
    \item \textbf{initramfs}: A temporary root filesystem loaded into memory during boot, allowing the kernel to initialize hardware, load necessary modules, and access the real root filesystem. It’s especially useful when you have encrypted filesystems, complex RAID configurations, or hardware that requires modules not built into the kernel.
\end{itemize}

% \begin{lstlisting}[caption={\smaller Sample Code}, numbers=left]
\begin{lstlisting}[caption={\smaller Sample Code}]
def hello_world():     ***#<------ this is a function***
    print("Hello, world!")
\end{lstlisting}

\begin{quote}
Focus on a single tree, and you'll miss the entire forest.
\end{quote}

\section{How They Work Together}
\begin{enumerate}
  \item \textbf{Kernel and Linux Firmware}:
    \begin{itemize}
        \item The kernel\ftn{This is the Kernel, I don't know what i'm doing, just need to make it longer} provides drivers to interact with hardware, but many devices (especially wireless cards, GPUs, etc.) require firmware that is not part of the kernel.
        \item If your hardware is supported by the kernel and doesn't require any additional proprietary firmware, you don’t need to worry about linux-firmware.
        \item But if you have hardware like certain Wi-Fi or Bluetooth cards, linux-firmware will be necessary to load the required binary blobs during boot\ftn{This is a footnote}.
    \end{itemize}
\item Kernel and initramfs:
    \begin{itemize}
        \item The kernel can directly access hardware if all drivers are built into it (no need for additional modules).
        \item If the hardware or system setup (e.g., encrypted partitions, RAID arrays, or non-standard root devices) needs additional modules, drivers, or other initializations, then initramfs will be required to load those things before the main filesystem is accessible.
        \item Without initramfs, the kernel might not be able to load certain drivers or modules that are required to boot, especially if those modules aren’t built directly into the kernel but are instead loaded at boot time.
    \end{itemize}
\end{enumerate}

\simplefigure{./assets/images/image}{Test Image 1}

\newpage

\section{Highly Customized Kernel}
If you \textit{customize your kernel} thoroughly by:
\begin{itemize}
  \item \textit{Including all necessary drivers directly in the kernel}, instead of as modules, and
  \item Ensuring that \textit{your hardware doesn't require any extra firmware},
\end{itemize}

then \textit{you can avoid both} \textbf{linux-firmware} and \textbf{initramfs}. The {\icode{kernel}} will handle all the device initialization on its own, without needing \textit{initramfs} to load modules or extra firmware at boot time.


\begin{lstlisting}[caption={\smaller Sample Code}]
elliot@archlinux ~ $ cat /etc/os-release                                                     [9:53:41]
NAME="Arch Linux"
PRETTY_NAME="Arch Linux"
ID=arch
BUILD_ID=rolling
ANSI_COLOR="38;2;23;147;209"
HOME_URL="https://archlinux.org/"
DOCUMENTATION_URL="https://wiki.archlinux.org/"
SUPPORT_URL="https://bbs.archlinux.org/"
BUG_REPORT_URL="https://gitlab.archlinux.org/groups/archlinux/-/issues"
PRIVACY_POLICY_URL="https://terms.archlinux.org/docs/privacy-policy/"
LOGO=archlinux-logo
\end{lstlisting}
